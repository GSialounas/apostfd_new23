%%%%%%%%%%%%%%%%%%%%%%%%%%%%%%%%%%%%%%%%%%%%%%%%%%%%%%%%%%%%%%%%%%%%%%%
%% Hey Emacs,
%% mode:latex ***
%% tex-main-file: "apostfd.tex"  ***
%%
%%        Reconstruction based a posteriori estimates for some finite difference schemes.
%%
%%
%% Dear Human:
%%
%% This source file may be subject to copyright restrictions as mandated
%% by the Journal's policy and arXiv's copyright policies.  
%%
%% In any case, you MUST SEEK PERMISSION from at least one of the authors
%% prior to any use of this source file.
%%
%%%%%%%%%%%%%%%%%%%%%%%%%%%%%%%%%%%%%%%%%%%%%%%%%%%%%%%%%%%%%%%%%%%%%%%%
%%%%%%%%%%%%%%%%%%%%%%%%%%%%%%%%%%%%%%%%%%%%%%%%%%%%%%%%%%%%%%%%%%%%%%%%
%% style and macro packages
%%%%%%%%%%%%%%%%%%%%%%%%%%%%%%%%%%%%%%%%%%%%%%%%%%%%%%%%%%%%%%%%%%%%%%%%

\documentclass[final]{amsart}
\usepackage{pdfsync}
\usepackage{amsmath,amssymb}
\usepackage{enumerate}
\usepackage{xspace}

\usepackage{boxedminipage}
\usepackage{algorithmicx}
\usepackage[ruled]{algorithm}
\usepackage{algpseudocode}
\usepackage{listings}% for including source code
\usepackage{tabularx}
\usepackage{subfigure}
\usepackage{tristan}
\usepackage{fullpage}
\usepackage[colorlinks,citecolor=cyan,linkcolor=magenta]{hyperref}

%\usepackage[cm]{fullpage}
\renewcommand{\bi}[2]{\ensuremath{\cA\qp{#1,#2}}}
\renewcommand{\bih}[2]{\ensuremath{\cA_h\qp{#1,#2}}}
\renewcommand{\bid}[2]{\ensuremath{\cA^*\qp{#1,#2}}}
\renewcommand{\biloc}[2]{\ensuremath{\cA^K\qp{#1,#2}}}
\renewcommand{\bihloc}[2]{\ensuremath{\cA_h^K\qp{#1,#2}}}


\renewcommand{\vec}[1]{\geovec{#1}}
\renewcommand{\mat}[1]{\geomat{#1}}

\renewcommand{\enorm}[2]{\ensuremath{\Norm{#1}}_{dG,{#2}}}
\newcommand{\opnorm}[2]{\ensuremath{\Norm{#1}}_{\widetilde{dG},{#2}}}
\renewcommand{\eenorm}[2]{\,|\!|\!| {{#1}} |\!|\!|_{dG,{#2}}}
\renewcommand{\d}{\ensuremath{\,\mathrm{d}}}
\renewcommand{\todo}[1]{{\color{red} Tristan says:  #1 }}

%%%%%%%%%%%%%%%%%%%%%%%%%%%%%%%%%%%%%%%%%%%%%%%%%%%%%%%%%%%%%%%%%%%%%%%%
%% local macros
%%%%%%%%%%%%%%%%%%%%%%%%%%%%%%%%%%%%%%%%%%%%%%%%%%%%%%%%%%%%%%%%%%%%%%%%

\numberwithin{equation}{section}
%%%%%%%%%%%%%%%%%%%%%%%%%%%%%%%%%%%%%%%%%%%%%%%%%%%%%%%%%%%%%%%%%%%%%%%%
%% service packages and switches
%%%%%%%%%%%%%%%%%%%%%%%%%%%%%%%%%%%%%%%%%%%%%%%%%%%%%%%%%%%%%%%%%%%%%%%%
%\usepackage{showkeys}
\setboolean{showtodo}{true}
\setboolean{shownotes}{true}
\setboolean{showchanges}{false}%true
\setboolean{usemathrsfs}{false}%true
\setlength{\parindent}{12pt} %to have no indentation in the beginning
                            %of paragraph
%\setboolean{issiamltex}{true}

%\usepackage{tikz}
%\usetikzlibrary{shapes,arrows}


\author{
  Tristan Pryer
}
\address{
  Tristan Pryer
  \thanks{
    Department of Mathematical Sciences,
    University of Bath, Bath BA2 7AY, UK
    {\tt{tmp38@bath.ac.uk}}.
}}

\author{
  Georgios Sialounas
}
\address{
  Georgios Sialounas
  \thanks{
    Swears like a sailor, fill this in!
}}

%%%%%%%%%%%%%%%%%%%%%%%%%%%%%%%%%%%%%%%%%%%%%%%%%%%%%%%%%%%%%%%%%%%%%%%% 
\title{A posteriori bound for Burgers' equation in the pre-schock regime}
\date{\today}
%%%%%%%%%%%%%%%%%%%%%%%%%%%%%%%%%%%%%%%%%%%%%%%%%%%%%%%%%%%%%%%%%%%%%%%%
\pdfformat{true}
%%%%%%%%%%%%%%%%%%%%%%%%%%%%%%%%%%%%%%%%%%%%%%%%%%%%%%%%%%%%%%%%%%%%%%%%

\begin{document}
\maketitle

%%%%%%%%%%%%%%%%%%%%%%%%%%%%%%%%%%%%%%%%%%%%%%%%%%%%%%%%%%%%%%%%%%%%%%%%

Let $u\qp{x,t}$ solve 
\begin{equation}
\begin{aligned}
u_t+\partial_x\qp{\frac{u^2}{2}}&=0\\
u\qp{x,0}&=u_0\qp{x}
\end{aligned}
\end{equation}
with periodic boundary conditions.  Let $v$ satisfy the perturbed pde
\begin{equation}
\begin{aligned}
v_t+\partial_x\qp{\frac{v^2}{2}}&=-R\\
v\qp{x,0}&=v_0\qp{x}.
\end{aligned}
\end{equation}
Then, the error $e:=u-v$ satisfies
\begin{equation}
\begin{aligned}
e_t+\partial_x\qp{\frac{\qp{u+v}\qp{u-v}}{2}}&=R\\
e\qp{x,0}&=e_0\qp{x}.
\end{aligned}
\end{equation}
We use the fact that $u+v = u+v-v+v = e+2v$ to rewrite this as
\begin{equation}
\begin{aligned}
e_t+\partial_x\qp{\frac{\qp{e+2v}\qp{e}}{2}}&=R\\
e\qp{x,0}&=e_0\qp{x}.
\end{aligned}
\end{equation}
We test with $e$ to obtain

\begin{equation}
\begin{aligned}
\int Re&=\int e e_t +\frac{1}{2}e\partial_x{\qp{e^2+2ve}}\\
\int Re&=\int e e_t +\int \frac{1}{3}\partial_x{\qp{e^3}}+\int e\partial_x{\qp{ve}}
\end{aligned}
\end{equation}
We note that 
\begin{equation}
.5\partial_x\qp{ve^2} = {vee_x}+.5{e^2v_x}
\end{equation}
and that
\begin{equation}
\begin{aligned}
 e\partial_x{\qp{ve}}&= {vee_x}+{e^2v_x}\\
&=.5\partial_x\qp{ve^2}+.5{e^2v_x}.
\end{aligned}
\end{equation}
Hence
\begin{equation}
\begin{aligned}
\frac{1}{2}\frac{\mathrm{d}}{dt}\Norm{e}^2_{L_2}=\int Re +\int.5 e^2 v_x
\end{aligned}
\end{equation}
We use the Cauchy-Schwarz and Holder's inequality to get
\begin{equation}
\begin{aligned}
\frac{1}{2}\frac{\mathrm{d}}{dt}\Norm{e}^2_{L_2}&\leq\Norm{R}_{L_2}\Norm{e} _{L_2}+.5 \Norm{e}^2_{L_2}\Norm{v_x}_{L_\infty}\\
&\leq\frac{1}{2}\Norm{R}^2_{L_2}+\frac{1}{2}\Norm{e}^2_{L_2}+\frac{1}{2} \Norm{e}^2_{L_2}\Norm{v_x}_{L_\infty}\\
\end{aligned}
\end{equation}
We now use Gronwall's inequality to get
\begin{equation}
\begin{aligned}
\Norm{e\qp{t}}^2_{L_2}\leq \exp\qp{\int_0^t\qp{\Norm{v_x\qp{s}}_{L_\infty}+1}\mathrm{d}s}\qb{\Norm{e\qp{0}}_{L_2}^2+\int_0^t\Norm{R\qp{s}}^2_{L_2}\mathrm{d}s}
\end{aligned}
\end{equation}





\bibliographystyle{alpha}
\bibliography{./tristansbib,./tristanswritings,./apostfd,./apostfd_bibdesk}

\end{document}

%% End: ***
    
